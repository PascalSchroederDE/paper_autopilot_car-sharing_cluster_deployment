%!TEX root = ../dokumentation.tex

\pagestyle{plain}

\iflang{en}{%
\addchap{Abstrakt - Deutsch} % Text für Überschrift

Im Jahr 2035 wird die Produktion laut Prognosen von autonomen Autos auf 48 Millionen pro Jahr angehoben werden.\textsuperscript{cmp.\cite{1}} Diese neue Technologie wird unseren Weg und unser Verständnis von Mobilität völlig verändern. Nicht jeder wird in Zukunft sein eigenes Auto haben - nicht nur wegen der autonomen Fahrzeuge, sondern auch wegen anderer Probleme wie der zunehmenden Urbanisierung und der damit verbundenen Parkplatznot sowie der Schadstoffproblematik.

Um diesem Wandel in der Welt der Mobilität zu begegnen, hat die EU-Kommission das Projekt AUTOPILOT ins Leben gerufen, das sich um das autonome Fahren, eine \acs{IoT}-Infrastruktur und zuletzt auch Dienstleistungen zur Nutzung dieser neuen Technologien und Möglichkeiten kümmern soll. Einer davon ist der im IBM Research Lab in Dublin entwickelte Carsharing-Service.

In dieser Arbeit werden das AUTOPILOT-Projekt sowie das Carsharing-Projekt vorgestellt und beschrieben. Auch die zugrunde liegenden Technologien des Carsharing-Dienstes werden aufgelistet und erläutert. Dies ist das Basiswissen für den Hauptteil dieser Arbeit - die Umstellung des Carsharing-Dienstes und all seiner Komponenten von einer lokalen Bereitstellung auf eine Bereitstellung innerhalb eines Clusters auf IBM Cloud. Die Art und Weise, wie dies geschehen kann, wird beschrieben und der Erfolg des Umzugs zum Cluster wird am Ende dieser Arbeit bewertet.

}