%!TEX root = ../dokumentation.tex

\chapter{Discussion}

The defined criteria in chapter 3.1 determined the goals of this project. In the following chapter those will be compared to its results and evaluated for its outcomes.

The first goal was to outsource the Kafka client from a local client to IBM Cloud Message Hub. For testing its functionality a web application had to be created, which can produce and consume messages. The results were described of this goal was described in chapter 4.1 and can be considered as successfull. The web app can produce messages of any topic and there are also consumer for the two predefined topics book and confirm, which enables an accurate testing of the Kafka client. This client was also moved from local docker containers running on a Zookeeper client to a completly independent IBM Cloud service calles "Message Hub". This enables an access to the Kafka client without the need of any local setup more than running the car sharing app.

The second goal was to deploy a DummyScheduler on a Kubernetes cluster hosted internally on IBM Cloud. The internal hosting caused some additional security restrictions. That's why it was part of the work to figure out the differences from deploying apps on this internal cluster compared to a "normal" Kubernetes cluster. For that some additional configurations for initializing the Kubectl client (cmp. chapter 3.2) were necessary. Also the docker container had to be uploaded to a internal container registry, which caused the need of creating secrets on the cluster for being able to pull those containers on the cluster. Last also the exposing of the services for external use is different and needs some more configurations in the ingress.yaml file (cmp. chapter 3.5???). 

All those differences could be figured out and were completly documented for the rest of the team. Also the DummyScheduler and its dependencies could be dockerized, which was a condition for deploying it on the cluster. This container could then be uploaded to the container registry and deployed on the cluster, which enabling the project to be considered successfull, because all the goals were fullfilled. The DummyScheduler is running on the cluster and all its functionalities are working, which can be tested with the Kafka test app described in chapter 4.1. 

Additional to deploying the DummyScheduler during this project also the simulation of the car sharing app, which includes the usage of the real scheduler, could be successfully deployed on the cluster. This needed the docker contaienr of the car sharing app to include a small part of the cplex engine. Additionally for that an extended docker container for each - the mongo database as well as the OSRM service - had to be created and deployed as described in chapter 3.x. All this work was successfull, which makes it possible to run a simulation on the cluster and visualize it with a locally running demo UI, which can be seen at the end of chapter 4.2.

All in all this additional deployments of more than just the DummyScheduler allowing the project to be considered as overachieved, because the goals were exceeded.